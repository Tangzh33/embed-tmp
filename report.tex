\documentclass[a4paper, 12pt]{article}
\usepackage{amsmath}
\usepackage{graphicx}
\usepackage{geometry}
\usepackage{listings}
\usepackage{xcolor}
\usepackage[UTF8]{ctex}
\usepackage{fancyhdr}
\usepackage{float}
\usepackage{bm}
\usepackage{amssymb}
\usepackage{enumerate}
\usepackage{enumitem}
% Template for pseudo-code
\usepackage{algorithm}
\usepackage{algorithmicx}
\usepackage{algpseudocode}
\usepackage{amsmath}
\floatname{algorithm}{算法}
\renewcommand{\algorithmicrequire}{\textbf{输入:}}
\renewcommand{\algorithmicensure}{\textbf{输出:}}

\setitemize[1]{itemsep=0pt,partopsep=0pt,parsep=\parskip,topsep=5pt}

% Preset for real code
\definecolor{dkgreen}{rgb}{0,0.6,0}
\definecolor{gray}{rgb}{0.5,0.5,0.5}
\definecolor{mauve}{rgb}{0.58,0,0.82}

% \lstset{frame=shadowbox,
%     language=c++,
%     aboveskip=3mm,
%     belowskip=3mm,
%     showstringspaces=false,
%     columns=flexible,
%     basicstyle={\small\ttfamily},
%     numbers=left,
%     numberstyle=\tiny\color{gray},
%     keywordstyle=\color{blue},
%     stringstyle=\color{mauve},
%     breaklines=true,
%     breakatwhitespace=true,
%     tabsize=3
% }
\lstdefinestyle{c++}{
   frame=shadowbox,
    language=c++,
    aboveskip=3mm,
    belowskip=3mm,
    showstringspaces=false,
    columns=flexible,
    basicstyle={\small\ttfamily},
    numbers=left,
    numberstyle=\tiny\color{gray},
    keywordstyle=\color{blue},
    stringstyle=\color{mauve},
    breaklines=true,
    breakatwhitespace=true,
    tabsize=3
}
\lstdefinestyle{bash}{
   frame=shadowbox,
    language=bash,
    aboveskip=3mm,
    belowskip=3mm,
    showstringspaces=false,
    columns=flexible,
    basicstyle={\small\ttfamily},
    numbers=left,
    numberstyle=\tiny\color{gray},
    keywordstyle=\color{blue},
    stringstyle=\color{mauve},
    breaklines=true,
    breakatwhitespace=true,
    tabsize=3
}
\lstdefinestyle{python}{
   frame=shadowbox,
    language=python,
    aboveskip=3mm,
    belowskip=3mm,
    showstringspaces=false,
    columns=flexible,
    basicstyle={\small\ttfamily},
    numbers=left,
    numberstyle=\tiny\color{gray},
    keywordstyle=\color{blue},
    stringstyle=\color{mauve},
    breaklines=true,
    breakatwhitespace=true,
    tabsize=3
}

%目录的颜色修改
\usepackage[colorlinks,linkcolor=black]{hyperref}
\geometry{scale=0.8}
\begin{document}
\thispagestyle{empty}
\pagestyle{fancy}
%\fancyhf{} % 清空当前的页眉页脚
\lhead{\small\leftmark}   %页眉左侧显示页数                 
\chead{嵌入式系统Project\ 1}                                  %页眉中
\rhead{\small\rightmark}                         %章节信息                       
\cfoot{\thepage}                                %当前页,记得调用前文提到的宏包                
%\rfoot{页脚左}                                                   
\renewcommand{\headrulewidth}{0.4pt} 
\renewcommand{\footrulewidth}{0.4pt}

\begin{center}
    
    \phantom{Start!}
	\vspace{2cm}
	
	\begin{figure}[ht]
	\includegraphics[width=400pt]{logo.png} 
	\end{figure}
	
    \center{
		\textbf{\zihao{1} 嵌入式系统}\\
		  \vspace{0.25cm}
       	  \textbf{\zihao{1} Project 1}\\
       	  \vspace{0.5cm}
          \textbf{\zihao{3} (2022 学年 秋 季 学期)}        
    }
    \vspace{3.5cm}
    \begin{table}[!hbp]
    \centering
	\setlength{\abovecaptionskip}{1pt}
    \renewcommand\arraystretch{1.5}
     	\begin{tabular}{|c|c|}
     		\hline
     		\large ~~课程名称~~ &\large ~~~~~~嵌入式系统~~~~~~ \\
     		\hline
     		\large 学院 &\large 计算机学院 \\
     		\hline
     		\large ~~授课教师~~ &\large 黄凯 \\
			\hline
			\large 学生姓名 &\large {\CJKfontspec{SIMSUN.TTC} 唐喆} \\
			\hline
			\large 学号 &\large 20337111 \\
			\hline
			\large 专业 &\large 计算机科学与技术 \\
			\hline
			\large Email &\large ~~~~~~tangzh\_33@163.com~~~~~~ \\
			\hline
			\large 完成时间 &  \multicolumn{1}{c|}{\large 2022年11月3日}\\
     		\hline
         \end{tabular}     		
     \end{table}
\end{center}

\title{	
    \normalfont \normalsize
    \textsc{School of Computer Science and Engineering, Sun Yat-sen University} \\ [25pt] %textsc small capital letters
    \rule{\textwidth}{0.5pt} \\[0.1cm] % Thin top horizontal rule
    \huge  嵌入式系统Homework\ 1\\ % The assignment title
    \rule{\textwidth}{2pt} \\[0.1cm] % Thick bottom horizontal rule
    \author{20337111   \CJKfontspec{SIMSUN.TTC}唐喆}
    \date{\normalsize 2022年11月3日}
}


\maketitle
\thispagestyle{empty}
\pdfbookmark[0]{目录}{toc}
\tableofcontents
\newpage


\section{论文阅读}
\subsection{A Model-Based Design Methodology for Cyber-Physical Systems}
\subsection{An Efficient Heuristic Procedure for Partitioning 
Graphs}


\section{死锁复现}
\subsection{源代码}
\subsection{An Efficient Heuristic Procedure for Partitioning 
Graphs}
% \section{Gem5 Hands-on}
% \subsection{Environment Setup and Build}

% \subsubsection{实验要求}
% Install gem5 on your own machines.

% \subsubsection{软硬件环境说明}
% 从掌握应用依赖、熟悉应用编译流程的角度出发,本次实验我选择在本机的WSL虚拟机\textbf{手动编译安装}。首先,本次实验的软硬件平台基本信息如下:
% \begin{table}[!ht]
%     \centering
%     \begin{tabular}{cc}
%     \hline
%         \textbf{Environment} & \textbf{Details} \\ \hline
%         CPU & 8核 \\ 
%         RAM & 16G \\ 
%         OS & WSL2-Ubuntu1804 \\ \hline
%     \end{tabular}
%     \caption{Software \& Hardware Environment}
% \end{table}

% \subsubsection{依赖安装与编译}
% \begin{itemize}
%     \item \textbf{依赖安装} \\
%     根据\textcolor{blue}{\href{https://www.gem5.org/documentation/general_docs/building}{官网教程}},我们可以知道依赖安装指令如下:
%     \begin{lstlisting}[style=bash]
% sudo apt-get update
% sudo apt install -y build-essential git m4 scons zlib1g zlib1g-dev \
%     libprotobuf-dev protobuf-compiler libprotoc-dev libgoogle-perftools-dev \
%     python3-dev python libboost-all-dev pkg-config
%     \end{lstlisting}
%     \item \textbf{源码下载} \\
%     谷歌源不能用,所以我使用Github的源代码:
%     \begin{lstlisting}[style=bash]
% git clone https://github.com/gem5/gem5.git
%     \end{lstlisting}
%     \item \textbf{安装} \\
%     根据官网教程和命令行提示,我们使用scons编译如下:
%     \begin{lstlisting}[style=bash]
% cd gem5
% /usr/bin/env python3 $(which scons) build/X86/gem5.opt -j8
%     \end{lstlisting}
% \end{itemize}

\end{document}